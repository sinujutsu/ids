% THIS IS SIGPROC-SP.TEX - VERSION 3.1
% WORKS WITH V3.2SP OF ACM_PROC_ARTICLE-SP.CLS
% APRIL 2009

\documentclass{acm_proc_article-sp}
%\usepackage{cite}

\begin{document}

\title{Network Intrusion Detection Proposal}

\numberofauthors{2} 
\author{
% 1st. author
\alignauthor
Erin Jamroz
       \affaddr{University of Puget Sound}\\
       \affaddr{1500 N. Alder St.}\\
       \affaddr{Tacoma, WA}\\
       \email{ejamroz@ups.edu}
% 2nd. author
\alignauthor
Jake Fuhrman
       \affaddr{University of Puget Sound}\\
       \affaddr{1500 N. Alder St.}\\
       \affaddr{Tacoma, WA}\\
       \email{jfuhrman@ups.edu}\\       
}

\maketitle
\begin{abstract}

\end{abstract}

\keywords{IDS: Intrusion Detection System\\
	      IPS: Intrusion Prevention System}

\section{Introduction}
    \subsection{Host-Based Audit Systems}
    \subsection{Network-Based Audit Systems}
    \subsection{Application-Based Audit Systems}
    \subsection{Signature Detection}
    \subsection{Anomaly Detection}

\section{Related Work}
    \subsection{Probes}
    \subsection{Privilege Escalation}
    \subsection{Denial of Service}

\section{Methods}
    % This is where we should talk about why we broke attacks up the way that we 
    % did.

\section{Evaluation}
    % I am not so sure we even need this section, if we use it we could talk 
    % about the limitations of IDS's of something, the current state of affairs

\section{Discussion}
    % What would we say here?

\section{Conclusions}
    % See outline for this bit

\section{References}
% The following two commands are all you need in the
% initial runs of your .tex file to
% produce the bibliography for the citations in your paper.
\bibliographystyle{abbrv}
\bibliography{Proposal}{}  % sigproc.bib is the name of the Bibliography in this case
\nocite{*}
% You must have a proper ".bib" file
%  and remember to run:
% latex bibtex latex latex
% to resolve all references
%
% ACM needs 'a single self-contained file'!
%
%APPENDICES are optional
%\balancecolumns
\balancecolumns
% That's all folks!
\end{document}
