% THIS IS SIGPROC-SP.TEX - VERSION 3.1
% WORKS WITH V3.2SP OF ACM_PROC_ARTICLE-SP.CLS
% APRIL 2009

\documentclass{acm_proc_article-sp}
%\usepackage{cite}

\begin{document}

\title{Network Intrusion Detection Proposal}

\numberofauthors{2} 
\author{
% 1st. author
\alignauthor
Erin Jamroz
       \affaddr{University of Puget Sound}\\
       \affaddr{1500 N. Alder St.}\\
       \affaddr{Tacoma, WA}\\
       \email{ejamroz@ups.edu}
% 2nd. author
\alignauthor
Jake Fuhrman
       \affaddr{University of Puget Sound}\\
       \affaddr{1500 N. Alder St.}\\
       \affaddr{Tacoma, WA}\\
       \email{jfuhrman@ups.edu}\\       
}

\maketitle
\begin{abstract}
This paper provides a description of the motivation, preliminary design strategy, and importance of designing a Network Intrusion Detection System. Living in an age of highly complex and coupled networks necessitates having tools to monitor and secure them. While there exist powerful open source solutions for network monitoring, we feel that the exercise of designing a rudimentary system from scratch will grant us highly valuable and extendable skills.
\end{abstract}

\keywords{IDS: Intrusion Detection System\\
			IPS: Intrusion Prevention System\\
			SVM: Support Vector Machine\\
			ANN: Artificial Neural Network\\ 
			AI: Artificial Intelligence}

\section{Introduction}

\section{Related Work}

\section{Methods}

\section{Evaluation}

\section{Discussion}

\section{Conclusions}


% The following two commands are all you need in the
% initial runs of your .tex file to
% produce the bibliography for the citations in your paper.
\bibliographystyle{abbrv}
\bibliography{Proposal}{}  % sigproc.bib is the name of the Bibliography in this case
\nocite{*}
% You must have a proper ".bib" file
%  and remember to run:
% latex bibtex latex latex
% to resolve all references
%
% ACM needs 'a single self-contained file'!
%
%APPENDICES are optional
%\balancecolumns

\balancecolumns
% That's all folks!
\end{document}
